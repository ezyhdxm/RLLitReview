\documentclass[letterpaper,11pt]{article}
\pagestyle{plain}

%\usepackage{showkeys}

\usepackage[top=1in,bottom=1.25in, left=1in, right=1in]{geometry}
\usepackage{hyperref}
\usepackage{amsmath}
\usepackage[font=footnotesize]{caption}
% The amsmath package sets \interdisplaylinepenalty to 10000
% thus preventing page breaks from occurring within multiline equations. Use:
\interdisplaylinepenalty=2500
% to restore such page breaks as IEEEtran.cls normally does.
\usepackage{algorithm,algorithmic}
\usepackage{amssymb,amsthm, mathrsfs}
\usepackage{array}
\usepackage{enumerate}
\usepackage{enumitem}
%\usepackage{enumitem-zref}
\usepackage{graphicx}
\usepackage[dvipsnames]{xcolor}
\usepackage{subfig}
\usepackage{tikz}
\usepackage{pgf}
\usepackage{booktabs}
\usetikzlibrary{arrows, positioning, patterns, patterns.meta}
\usepackage{tikz-cd}
\usepackage{natbib}
\usepackage{bbm}
\bibpunct{(}{)}{;}{a}{}{,}
\usepackage{stackengine}
\usepackage{pifont}
\usepackage{mathabx}

\title{Multi-Armed Bandit and Linear Bandit}
\author{Hao Yan}
%\date{March 2023}
\date{}

\begin{document}
\newtheorem{theorem}{Theorem}
\newtheorem{lemma}{Lemma}
\newtheorem{proposition}{Proposition}
\newtheorem{observation}{Observation}
\newtheorem{remark}{Remark}
\newtheorem{example}{Example}
\newtheorem{definition}{Definition}
\newtheorem{corollary}{Corollary}
\newtheorem{assumption}{Assumption}

\renewcommand{\AA}[1]{A^{(#1)}}
\newcommand{\HH}[1]{H^{(#1)}}
\newcommand{\nub}[1]{(\nu_{#1} + b_{#1}^2)}

\newcommand{\ssigma}[1]{\rho^{(#1)}}
\newcommand{\mxbernsymbol}{\eta}

%%%%%%%%%%
% mathbb %
%%%%%%%%%%

\renewcommand{\Pr}{\mathbb{P}}
\newcommand{\E}{\mathbb{E}}
\newcommand{\N}{\mathbb{N}}
\newcommand{\R}{\mathbb{R}}
\newcommand{\Z}{\mathbb{Z}}
\newcommand{\bbA}{\mathbb{A}}
\newcommand{\bbB}{\mathbb{B}}
\newcommand{\bbC}{\mathbb{C}}
\newcommand{\bbD}{\mathbb{D}}
\newcommand{\bbF}{\mathbb{F}}
\newcommand{\bbG}{\mathbb{G}}
\newcommand{\bbH}{\mathbb{H}}
\newcommand{\bbI}{\mathbb{I}}
\newcommand{\bbJ}{\mathbb{J}}
\newcommand{\bbK}{\mathbb{K}}
\newcommand{\bbL}{\mathbb{L}}
\newcommand{\bbM}{\mathbb{M}}
\newcommand{\bbO}{\mathbb{O}}
\newcommand{\bbQ}{\mathbb{Q}}
\newcommand{\bbS}{\mathbb{S}}
\newcommand{\bbT}{\mathbb{T}}
\newcommand{\bbU}{\mathbb{U}}
\newcommand{\bbV}{\mathbb{V}}
\newcommand{\bbW}{\mathbb{W}}
\newcommand{\bbX}{\mathbb{X}}
\newcommand{\bbY}{\mathbb{Y}}
\newcommand{\bbone}{\mathbbm{1}}



\newcommand{\Rnonneg}{\R_{\ge 0}}

\newcommand{\calA}{\mathcal{A}}
\newcommand{\calB}{\mathcal{B}}
\newcommand{\calC}{\mathcal{C}}
\newcommand{\calD}{\mathcal{D}}
\newcommand{\calE}{\mathcal{E}}
\newcommand{\calF}{\mathcal{F}}
\newcommand{\calG}{\mathcal{G}}
\newcommand{\calH}{\mathcal{H}}
\newcommand{\calI}{\mathcal{I}}
\newcommand{\calJ}{\mathcal{J}}
\newcommand{\calK}{\mathcal{K}}
\newcommand{\calL}{\mathcal{L}}
\newcommand{\calM}{\mathcal{M}}
\newcommand{\calN}{\mathcal{N}}
\newcommand{\calP}{\mathcal{P}}
\newcommand{\calQ}{\mathcal{Q}}
\newcommand{\calR}{\mathcal{R}}
\newcommand{\calS}{\mathcal{S}}
\newcommand{\calT}{\mathcal{T}}
\newcommand{\calU}{\mathcal{U}}
\newcommand{\calV}{\mathcal{V}}
\newcommand{\calW}{\mathcal{W}}
\newcommand{\calX}{\mathcal{X}}
\newcommand{\calY}{\mathcal{Y}}

\newcommand{\Ahat}{\hat{A}}
\newcommand{\Ihat}{\hat{I}}
\newcommand{\Ohat}{\hat{O}}
\newcommand{\Phat}{\hat{P}}
\newcommand{\Rhat}{\hat{R}}
\newcommand{\Xhat}{\hat{X}}
\newcommand{\dhat}{\hat{d}}
\newcommand{\uhat}{\hat{u}}
\newcommand{\betahat}{\hat{\beta}}
\newcommand{\muhat}{\hat{\mu}}
\newcommand{\thetahat}{\hat{\theta}}
\newcommand{\what}{\hat{w}}
\newcommand{\rhohat}{\hat{\rho}}
\newcommand{\sigmahat}{\hat{\sigma}}
\newcommand{\tauhat}{\hat{\tau}}
\newcommand{\nuhat}{\hat{\nu}}
\newcommand{\chat}{\hat{c}}
\newcommand{\bhat}{\hat{b}}
\newcommand{\rhat}{\hat{r}}
\newcommand{\Zhat}{\widehat{Z}}

\newcommand{\ccheck}{\check{c}}
\newcommand{\Xcheck}{\check{X}}
\newcommand{\Aml}{\mathring{A}}
\newcommand{\Vml}{\mathring{V}}
\newcommand{\Xml}{\mathring{X}}
\newcommand{\wml}{\mathring{w}}

\newcommand{\Abar}{\bar{A}}
\newcommand{\Pbar}{\bar{P}}
\newcommand{\Xbar}{\bar{X}}
\newcommand{\cbar}{\bar{c}}
\newcommand{\Cbar}{\bar{C}}
\newcommand{\Wbar}{\bar{W}}
\newcommand{\mCbar}{\bar{\mC}}
\newcommand{\mWbar}{\bar{\mW}}
\newcommand{\nubar}{\bar{\nu}}
\newcommand{\bbar}{\bar{b}}
\newcommand{\vbar}{\bar{v}}
\newcommand{\sigbar}{\bar{\sigma}}

\newcommand{\Atilde}{\widetilde{A}}
\newcommand{\Deltatil}{\widetilde{\Delta}}
\newcommand{\UAtilde}{U_{\Atilde}}
\newcommand{\SAtilde}{S_{\Atilde}}
\newcommand{\Wtilde}{\widetilde{W}}
\newcommand{\Ptilde}{\widetilde{P}}
\newcommand{\Xtilde}{\widetilde{X}}
\newcommand{\dtilde}{\tilde{d}}
\newcommand{\wtilde}{\tilde{w}}
\newcommand{\rhotilde}{\tilde{\rho}}
\newcommand{\calDtilde}{\tilde{\calD}}
\newcommand{\calXtilde}{\tilde{\calX}}

\newcommand{\dH}{d_{\text{H}}}
\newcommand{\dHtilde}{\dtilde_{\text{H}}}


\newcommand{\Xstar}{X^\star}
\newcommand{\bstar}{b^\star}
\newcommand{\lambdastar}{\lambda^\star}
\newcommand{\mLambdastar}{\mLambda^\star}
\newcommand{\nustar}{\nu^\star}
\newcommand{\kd}{{k,\cdot}}
\newcommand{\mut}{\tilde{\mu}}
\newcommand{\nmin}{n_{\min}}

\newcommand{\mUstart}{\mU^{\star \top}}
\newcommand{\ld}{{\ell,\cdot}}
\newcommand{\mUl}{\mU_\ld}
\newcommand{\mMl}{\mM_\ld}
\newcommand{\mAl}{\mA_\ld}
\newcommand{\mUstarl}{\mUstar_\ld}
\newcommand{\mMstarl}{\mMstar_\ld}
\newcommand{\mWl}{\mW_\ld}


\newcommand{\mMhat}{\widehat{\mM}}
\newcommand{\mBhat}{\widehat{\mB}}

\newcommand{\mUll}{\mU^{(\ell)}}
\newcommand{\mUld}{\mU^{(\ell)}_{\ell, \cdot}}
\newcommand{\mHll}{\mH^{(\ell)}}
\newcommand{\mMll}{\mM^{(\ell)}}

\newcommand{\wopt}{\wml}

\newcommand{\SA}{S_A}
\newcommand{\SM}{S_M}
\newcommand{\SP}{S_P}
\newcommand{\UP}{U_P}
\newcommand{\UA}{U_A}
\newcommand{\UM}{U_M}

\newcommand{\RDPG}{\operatorname{RDPG}}
\newcommand{\GRDPG}{\operatorname{GRDPG}}
\newcommand{\ASE}{\operatorname{ASE}}
\newcommand{\KL}{\operatorname{KL}}
\newcommand{\SNR}{\operatorname{SNR}}
\newcommand{\VAR}{\operatorname{Var}}
\newcommand{\COV}{\operatorname{Cov}}
\newcommand{\wtd}{\operatorname{wtd}}
\newcommand{\unif}{\operatorname{unif}}
\newcommand{\Bernoulli}{\operatorname{Bern}}
\newcommand{\supp}{\operatorname{supp}}
\newcommand{\rank}{\operatorname{rank}}
\newcommand{\diag}{\operatorname{diag}}
\newcommand{\tr}{\operatorname{tr}}

\newcommand{\dtildetti}{\dtilde_{\tti}}

%\newcommand{\Phatwtd}{\Phat_{\wtd}}
\newcommand{\Phatwtd}{\Ptilde}
%\newcommand{\Phatunif}{\Phat_{\unif}}
\newcommand{\Phatunif}{\Pbar}

\newcommand{\onevec}{\vec{\boldsymbol{1}}}
\newcommand{\ivec}{\vec{i}}
\newcommand{\jvec}{\vec{j}}
\newcommand{\indicator}{\mathbb{I}}
\newcommand{\indic}{\indicator}

\newcommand{\marginal}[1]{\marginpar{\raggedright\scriptsize #1}}
\newcommand{\as}{\text{~~~a.s.}}
\newcommand{\whp}{\text{~~~w.h.p.}}
\newcommand{\inlaw}{\xrightarrow{\calL}}
\newcommand{\inprob}{\xrightarrow{P}}
\newcommand{\iid}{\stackrel{\text{i.i.d.}}{\sim}}
\newcommand{\eqdist}{\stackrel{d}{=}}
\newcommand{\Perm}{\Pi}

%%%%%%%%%%%%%%%%%%%%%%%%%%%%%%%%%%%%%
%%% New commands for this project %%%
%%%%%%%%%%%%%%%%%%%%%%%%%%%%%%%%%%%%%

\newcommand{\vone}{\boldsymbol{1}}
\newcommand{\mA}{\boldsymbol{A}}
\newcommand{\mB}{\boldsymbol{B}}
\newcommand{\mC}{\boldsymbol{C}}
\newcommand{\mD}{\boldsymbol{D}}
\newcommand{\mE}{\boldsymbol{E}}
\newcommand{\mF}{\boldsymbol{F}}
\newcommand{\mG}{\boldsymbol{G}}
\newcommand{\mH}{\boldsymbol{H}}
\newcommand{\mI}{\boldsymbol{I}}
\newcommand{\mJ}{\boldsymbol{J}}
\newcommand{\mK}{\boldsymbol{K}}
\newcommand{\mM}{\boldsymbol{M}}
\newcommand{\mO}{\boldsymbol{O}}
\newcommand{\mP}{\boldsymbol{P}}
\newcommand{\mQ}{\boldsymbol{Q}}
\newcommand{\mR}{\boldsymbol{R}}
\newcommand{\mS}{\boldsymbol{S}}
\newcommand{\mU}{\boldsymbol{U}}
\newcommand{\mV}{\boldsymbol{V}}
\newcommand{\mW}{\boldsymbol{W}}
\newcommand{\mX}{\boldsymbol{X}}
\newcommand{\mY}{\boldsymbol{Y}}
\newcommand{\mZ}{\boldsymbol{Z}}
\newcommand{\mLambda}{\boldsymbol{\Lambda}}
\newcommand{\mTheta}{\boldsymbol{\Theta}}
\newcommand{\mGamma}{\boldsymbol{\Gamma}}
\newcommand{\mPi}{\boldsymbol{\Pi}}
\newcommand{\mXhat}{\hat{\boldsymbol{X}}}
\newcommand{\mBstar}{\mB^\star}
\newcommand{\Bstar}{B^\star}
\newcommand{\Zstar}{Z^\star}
\newcommand{\mMstar}{\mM^\star}

\newcommand{\vx}{\boldsymbol{x}}
\newcommand{\vs}{\boldsymbol{s}}
\newcommand{\vy}{\boldsymbol{y}}
\newcommand{\ve}{\boldsymbol{e}}
\newcommand{\vu}{\boldsymbol{u}}
\newcommand{\vv}{\boldsymbol{v}}
\newcommand{\vo}{\boldsymbol{0}}
\newcommand{\vw}{\boldsymbol{w}}
\newcommand{\va}{\boldsymbol{a}}
\newcommand{\vz}{\boldsymbol{z}}
\newcommand{\vm}{\boldsymbol{m}}
\newcommand{\valpha}{\boldsymbol{\alpha}}
\newcommand{\vtheta}{\boldsymbol{\theta}}
\newcommand{\vsigma}{\boldsymbol{\sigma}}
\newcommand{\vgamma}{\boldsymbol{\gamma}}
\newcommand{\vbeta}{\boldsymbol{\beta}}
\newcommand{\veta}{\boldsymbol{\eta}}
\newcommand{\vlambda}{\boldsymbol{\lambda}}
\newcommand{\tmU}{\tilde{\mU}}
\newcommand{\tmV}{\tilde{\mV}}
\newcommand{\mSig}{\boldsymbol{\Sigma}}
\newcommand{\mDelta}{\boldsymbol{\Delta}}
\newcommand{\tmSig}{\tilde{\boldsymbol{\Sigma}}}
\newcommand{\tmLambda}{\tilde{\boldsymbol{\Lambda}}}
\newcommand{\tildV}{\tilde{V}}
\newcommand{\Opq}{\mathcal{O}_{p,q}}
\newcommand{\Od}{\mathbb{O}_d}
\newcommand{\tB}{\mathcal{B}}
\newcommand{\bM}{\mathbb{M}}
\newcommand{\bU}{\mathbb{U}}
\newcommand{\bV}{\mathbb{V}}
\newcommand{\vb}{\boldsymbol{b}}
\newcommand{\vdelta}{\boldsymbol{\delta}}
\newcommand{\vnu}{\boldsymbol{\nu}}
\newcommand{\I}{{I_\star}}
\newcommand{\mXi}{\boldsymbol{\Xi}}
\newcommand{\mUstar}{{\mU^\star}}
\newcommand{\Ustar}{{U^\star}}
\newcommand{\mPhi}{\boldsymbol{\Phi}}

\newcommand{\gsup}{S_{\mathcal{G}}}
\newcommand{\cG}{\mathcal{G}}
\newcommand{\zvec}{\boldsymbol{0}}
\newcommand{\veps}{\boldsymbol{\varepsilon}}
\newcommand{\vepsilon}{\boldsymbol{\epsilon}}
\newcommand{\vDelta}{\boldsymbol{\Delta}}
\newcommand{\sgn}[1]{\operatorname{sgn}(#1)}
\newcommand{\tti}[1]{\|#1\|_{2,\infty}}
\newcommand{\bora}{\hat{\vbeta}^{o}}
\newcommand{\thora}{\hat{\vtheta}^{o}}
\newcommand{\vora}{\hat{\vv}^{o}}
\newcommand{\gevent}{\mathbb{G}(\lambda_n)}
\newcommand{\mustar}{\mu^\star}
\newcommand{\conev}{\bbC_{\vv^*}(\bbM, \bbM^{\perp})}

\newcommand{\wDelta}{\widetilde{\mDelta}}
\newcommand{\wGamma}{\widetilde{\mGamma}}
\newcommand{\wW}{\widetilde{\mW}}

\newcommand{\inner}[1]{\langle #1 \rangle}
\newcommand{\sI}{\mathcal{I}}


\newcommand{\mHl}{\mH^{(\ell)}}
\newcommand{\mEl}{\mE^{(\ell)}}
\newcommand{\lDelta}{\mDelta^{(\ell)}}
\newcommand{\lLambda}{\mLambda^{(\ell)}}
\newcommand{\llambda}{\lambda^{(\ell)}}
\newcommand{\mZhat}{\widehat{\mZ}}
\newcommand{\mZtilde}{\widetilde{\mZ}}
\newcommand{\mZstar}{\mZ^\star}
\newcommand{\mPsi}{\boldsymbol{\Psi}}

\newcommand{\F}{\mathrm{F}}
\newcommand{\SDP}{\mathrm{SDP}}


\newcommand{\mYtil}{\widetilde{\mY}}
\newcommand{\mDeltatil}{\widetilde{\mDelta}}
\newcommand{\mWtil}{\widetilde{\mW}}
\newcommand{\Ytil}{\widetilde{Y}}



\newcommand{\vthetastar}{\vtheta^\star}
\newcommand{\thetastar}{\theta^\star}
\newcommand{\vthetastart}{\vtheta^{\star \top}}
\newcommand{\Mstar}{M^{\star}}

\newcommand{\psihat}{\widehat{\psi}}
\newcommand{\astar}{a^\star}
\newcommand{\regret}{\mathrm{regret}}
\newcommand{\LinUCB}{\texttt{LinUCB}}
\newcommand{\xstar}{x^\star}
\newcommand{\xstart}{x^{\star \top}}
\newcommand{\BALL}{\operatorname{BALL}}
\newcommand{\Astar}{A^\star}
\newcommand{\Sigstar}{\Sigma^\star}
\newcommand{\rstar}{r^\star}
\newcommand{\thetastart}{\theta^{\star \top}}
\maketitle

\section{Thompson sampling for two-armed bandit}


\begin{table}[h]
    \centering
    \renewcommand{\arraystretch}{1.2} % Adjust row height
    \begin{tabular}{c p{12cm}}
        \hline
        \textbf{Notations} & \textbf{Description} \\ 
        \hline
        $L$ & $L=24(\log T) / \Delta^2$.\\
        $j_0$ & Number of plays of the first arm until $L$ plays of the second arm.\\ 
        $t_j$ & Time step at which the $j$th play of the first arm happens. \\ 
        $Y_j$ & $Y_j = t_{j+1} - t_j - 1$ measures the number of time steps between the $j^{\text {th }}$ and $(j+1)^{\text {th }}$ plays of the first arm. \\ 
        $s(j)$ & Number of successes in the first $j$ plays of the first arm. \\ 
        $k_2(T)$ & Number of plays of the second arm in time $T$. \\
        $X(j, s, y)$ & Number of trials before a $\operatorname{Beta}(s+1, j-s+1)$ distributed random variable exceeds a threshold $y$.
        \\ \hline
    \end{tabular}
    \caption{Notation table}
    \label{tab:notation}
\end{table}


Here, we reproduce the proof of the two-armed bandit case. 
We will first bound the expectation of $Y_j$. 
The reason we want to control this term is because we have
\begin{equation*}
    \E [k_2(T)] \leq L + \E \left[\sum_{j=j_0}^{T-1} Y_j\right]
\end{equation*}
and 
\begin{equation*}
    \operatorname{Regret}(T) = k_2(T) \cdot \Delta. 
\end{equation*}
Bounding the above term controls the regret. 

By definition, $Y_j$ is the number of steps before $\theta_1(t) > \theta_2(t)$ after the $j^{\text{th}}$ play of the first arm. 
Thus, $Y_j$ can be written as
\begin{equation*}
    Y_j = \sum_{t = t_j+1}^T \prod_{s=t_{j}+1}^t \mathbbm{1}\left(\theta_1(s) \leq \theta_2(s)\right).
\end{equation*}
Consider the number of steps before $\theta_1(t) > \mu_2 + \frac{\Delta}{2}$ happens for the first time after the $j$th play of the first arm. 
Given $s(j)$, this has the same distribution as $X(j, s(j), \mu_2 + \frac{\Delta}{2})$. 
The term $Y_2$ can be decomposed into
\begin{equation*}
    \sum_{t = t_j+1}^T \prod_{s=t_{j}+1}^{t-1} \mathbbm{1}\left(\theta_1(s) \leq \theta_2(s)\right) \left[ \mathbbm{1}\left(\theta_1(t) > \mu_2 + \frac{\Delta}{2}, \theta_2(t) < \theta_1(t)\right) + \mathbbm{1}\left(\theta_1(t) > \mu_2 + \frac{\Delta}{2}, \theta_2(t) \geq \theta_1(t)\right)\right],
\end{equation*}
which is upper bounded by
\begin{equation*}
    \sum_{t = t_j+1}^T \prod_{s=t_{j}+1}^{t-1} \mathbbm{1}\left(\theta_1(s) \leq \theta_2(s)\right) \left[ \mathbbm{1}\left(\theta_1(t) > \mu_2 + \frac{\Delta}{2}\right) + \mathbbm{1}\left(\theta_2(t) > \mu_2 + \frac{\Delta}{2}\right)\right],
\end{equation*}
and the first part of the sum is given by 
\begin{equation*}
    X\left(j, s(j), \mu_2 + \frac{\Delta}{2}\right) \wedge T. 
\end{equation*}
Therefore, we have
\begin{equation*}
    Y_j \leq X\left(j, s(j), \mu_2 + \frac{\Delta}{2}\right) \wedge T + \sum_{t=t_j+1}^T \mathbbm{1}\left(\theta_2(t) > \mu_2 + \frac{\Delta}{2}\right). 
\end{equation*}
Summing over $j$ from $j_0$ to $T-1$, we have
\begin{equation*}
\begin{aligned}
    \sum_{j=j_0}^{T-1} Y_j &\leq \sum_{j=0}^{T-1} X\left(j, s(j), \mu_2 + \frac{\Delta}{2}\right) \wedge T + \sum_{j=j_0}^{T-1} \sum_{t=t_j+1}^T \mathbbm{1}\left(\theta_2(t) > \mu_2 + \frac{\Delta}{2}, j \geq j_0\right)\\
    &\leq \sum_{j=0}^{T-1} X\left(j, s(j), \mu_2 + \frac{\Delta}{2}\right) \wedge T + T \cdot \sum_{t=1}^T \mathbbm{1}\left(\theta_2(t) > \mu_2 + \frac{\Delta}{2}, k_2(t) \geq L\right),
\end{aligned}
\end{equation*}
where the last inequality follows from the definition that any $j \geq j_0$, we must have $t \geq t_j$ to satisfy $k_2(t) \geq L$. 
Taking expectation on both sides, we have
\begin{equation*}
    \E \left[\sum_{j=j_0}^{T-1} Y_j\right] \leq \sum_{j=0}^{T-1} \E \left[ X\left(j, s(j), \mu_2 + \frac{\Delta}{2}\right) \wedge T \right] + T \cdot \sum_{t=1}^T \Pr\left(\theta_2(t) > \mu_2 + \frac{\Delta}{2}, k_2(t) \geq L\right).
\end{equation*}

Bounding the sum of probabilities is straightforward and we omit the details.
It remains to bound the first expectation. 
Conditioned on $s(j)$, we have
\begin{equation*}
\begin{aligned}
    \E \left[ X\left(j, s(j), y\right) \wedge T \mid s(j)\right] &\leq \E \left[ X\left(j, s(j), y\right) \mid s(j)\right] \wedge T\\
    &= \left[\frac{1}{F^B_{j+1, y}(s(j))} - 1\right] \wedge T. 
\end{aligned}
\end{equation*}
For large $j$, $s(j)$ can be bounded below using concentration inequality.
For small $j$, one explicitly bound 
\begin{equation*}
    \E \left[\frac{1}{F^B_{j+1, y}(s(j))} - 1\right] = \sum_{s=0}^{j} \frac{f_{j,\mu_1}^B(s)}{F_{j+1, y}^B(s)} - 1 = \sum_{s=\lceil y(j+1)\rceil}^j \frac{f_{j, \mu_1}^B(s)}{F_{j+1, y}^B(s)} + \sum_{s=0}^{\lceil y(j+1)\rceil-1} \frac{f_{j, \mu_1}^B(s)}{F_{j+1, y}^B(s)} - 1,
\end{equation*}
where $f_{j,\mu_1}^B$ denotes the pmf of the $\operatorname{Binomial}(j, \mu_1)$ distribution. We omit the details here, as it is mostly ad hoc calculations. 

Following the above derivation, we eventually arrive at the bound
$$
\mathbb{E}[\mathcal{R}(T)]=\mathbb{E}\left[\Delta \cdot k_2(T)\right] \leq\left(\frac{40 \ln T}{\Delta}+\frac{48}{\Delta^3}+18 \Delta\right)
$$

\newpage

\section{Thompson sampling for \texorpdfstring{$N$}{N}-armed bandit}

Before we dive into the proof, I want to remark that the following two papers on Thompson sampling are some of the worst written papers I have ever read. 

\begin{table}[h]
    \centering
    \renewcommand{\arraystretch}{1.2} % Adjust row height
    \begin{tabular}{c p{12cm}}
        \hline
        \textbf{Notations} & \textbf{Description} \\ 
        \hline
        $I_j$ & The interval between the $j$th and $(j+1)$th plays of the first arm. \\
        $M(t)$ & The event $M(t): \theta_1(t)>\max _{i \in C(t)} \mu_i+\frac{\Delta_i}{2}$
        \\ 
        $\gamma_j$ & $\left|\left\{t \in I_j: \mathbbm{1}(M(t)) = 1\right\}\right|$. \\
        $I_j(\ell)$ & The sub-interval of $I_j$ between the $(\ell-1)$th and $\ell$th occurrences of event $M(t)$ in $I_j$. \\
        $V_j^{\ell, a}$ & Number of steps in $I_j(\ell)$, for which $a$ is the best saturated arm, i.e., $V_j^{\ell, a}=\left|\left\{t \in I_j(\ell): \mu_a=\max _{i \in C(t)} \mu_i\right\}\right|$. \\
        
        \hline
    \end{tabular}
    \caption{Notation table}
    \label{tab:notation2}
\end{table}

In this proof, we introduce the concept of saturated and unsaturated arms. An arm $i \neq 1$ is in the saturated set $C(t)$ at time $t$, if it has been played at least $L_i := \frac{24 \log T}{\Delta_i^2}$ times before time $t$. 
For the set of saturated arms, it is not hard to show that their $\theta_i(t)$ tightly concentrated around their means, namely, 
$$
E(t): \quad\left\{\theta_i(t) \in\left[\mu_i-\Delta_i / 2, \mu_i+\Delta_i / 2\right], \forall i \in C(t)\right\} .
$$
holds with high probability. 
Regrets from the arms in the unsaturated set can be bound easily, since they get selected no more than $L_i$ times. 

At step $t$, a saturated arm $i$ can be played only if $\theta_i(t) \geq \theta_1(t)$. 
\begin{itemize}
    \item When $M(t)$ holds, the saturated arm $i$ can be played only if $\theta_i(t) > \mu_i + \Delta_i / 2$, namely, the event $E(t)$ does not hold.
    \item Otherwise, the saturated arm can be played and the number of plays is upper bounded by $\sum_{\ell=1}^{\gamma_j+1} |I_j(\ell)|$. 
\end{itemize}
 Thus, unless the high probability event $E(t)$ is violated, $M(t)$ denotes a play of an unsaturated arm at time $t$, and $\gamma_j$ essentially denotes the number of plays of unsaturated arms in interval $I_j$. The number of plays of saturated arms in interval $I_j$ is then at most 
\begin{equation*}
    \sum_{\ell=1}^{\gamma_j+1} \left|I_j(\ell)\right| + \sum_{t \in I_j} \mathbbm{1}\left(E(t)^c\right). 
\end{equation*}
Consider the case when $C(t)$ is nonempty otherwise, the event $M(t)$ always holds. The interval $I_j(\ell)$ is covered by $V_{j}^{\ell, a}$ and we have
\begin{equation*}
    |I_j(\ell)| = \sum_{a=2}^N V_j^{\ell, a}
\end{equation*}
If a saturated arm $i$ is played at time $t$ among one of the $V_j^{\ell,a}$ steps, then, either $E(t)$ is violated, or 
$$
\begin{aligned}
&\mu_i+\Delta_i / 2 \geq \theta_i(t) \geq \theta_a(t) \geq \mu_a-\Delta_a / 2
\end{aligned}
$$
which implies that
$$
\begin{aligned}
&\Delta_i=\mu_1-\mu_i \leq \mu_1-\mu_a+\frac{\Delta_a}{2}+\frac{\Delta_i}{2} \Rightarrow \Delta_i \leq 3 \Delta_a .
\end{aligned}
$$












\section{Thompson Sampling for Contextual Bandits with Linear Payoffs}

We start with a few concentration inequalities. 
Define $E^{\mu}(t)$ as the event that 
\begin{equation*}
    \forall i: \left|b_i(t)^\top \hat{\mu}(t) - b_i(t)^\top \mu\right| \leq \left[R \sqrt{d \log \left(\frac{t^3}{\delta}\right)} + 1\right] \sqrt{b_i(t)^\top B(t)^{-1} b_i(t)}. 
\end{equation*}
Recall that 
\begin{equation*}
    B(t) = I_d + \sum_{\tau=1}^{t-1} b_{a(\tau)}(\tau) b_{a(\tau)}(\tau)^\top 
\end{equation*}
and 
\begin{equation*}
\begin{aligned}
    \muhat(t) &= B(t)^{-1} \left(\sum_{\tau=1}^{t-1} b_{a(\tau)}(\tau) r_{a(\tau)}(\tau)\right) \\
    &= B(t)^{-1} \left(\sum_{\tau=1}^{t-1} b_{a(\tau)}(\tau) b_{a(\tau)}(\tau)^\top \mu\right) + B(t)^{-1} \left(\sum_{\tau=1}^{t-1} b_{a(\tau)}(\tau) \eta_{a(\tau), \tau}\right)\\
    &= \mu - B(t)^{-1} \mu + B(t)^{-1} \left(\sum_{\tau=1}^{t-1} b_{a(\tau)}(\tau) \eta_{a(\tau), \tau}\right). 
\end{aligned}
\end{equation*}
Therefore, we have
\begin{equation*}
\begin{aligned}
    \left|b_i(t)^\top \hat{\mu}(t) - b_i(t)^\top \mu\right| &\leq \left|b_i(t)^\top B(t)^{-1} \mu \right| + \left|b_i(t)^\top B(t)^{-1} \sum_{\tau=1}^{t-1} b_{a(\tau)}(\tau) \eta_{a(\tau), \tau}\right| \\
    &\leq \left\|b_i(t)\right\|_{B(t)^{-1}} \left\|\mu\right\|_{B(t)^{-1}} + \left\|b_i(t)\right\|_{B(t)^{-1}} \left\|\sum_{\tau=1}^{t-1} b_{a(\tau)}(\tau) \eta_{a(\tau), \tau}\right\|_{B(t)^{-1}}. 
\end{aligned}
\end{equation*}
Noting that 
\begin{equation*}
    \left|\mu^\top B(t)^{-1} \mu\right| \leq \|\mu\|_2 \left\| B(t)^{-1} \right\| \leq \|\mu\|_2 \leq 1, 
\end{equation*}
we have
\begin{equation*}
\begin{aligned}
    \left|b_i(t)^\top \hat{\mu}(t) - b_i(t)^\top \mu\right| &\leq \left\|b_i(t)\right\|_{B(t)^{-1}} + \left\|b_i(t)\right\|_{B(t)^{-1}} \left\|\sum_{\tau=1}^{t-1} b_{a(\tau)}(\tau) \eta_{a(\tau), \tau}\right\|_{B(t)^{-1}}. 
\end{aligned}
\end{equation*}
Taking $m_{\tau}$ to be $b_{a(\tau)}(\tau)$ and $\eta_{\tau}$ to be $\eta_{a(\tau), \tau}$ and applying Lemma~\ref{lem:self-norm}, we have
\begin{equation*}
    \left\|\sum_{\tau=1}^{t-1} b_{a(\tau)}(\tau) \eta_{a(\tau), \tau}\right\|_{B(t)^{-1}} \leq R \sqrt{d \log \left(\frac{t}{\delta}\right)}
\end{equation*}
holds with probability at least $1 - \delta$. 
Thus, it follows that with probability at least $1 - \delta/t^2$, the event $E^\mu(t)$ holds. 

Define $E^{\theta}(t)$ as the event that 
\begin{equation*}
    \forall i:\left|b_i(t)^\top \tilde{\mu}(t)-b_i(t)^\top \hat{\mu}(t)\right| \leq \min \left\{\sqrt{4 d \log (t)}, \sqrt{4 \log (t N)}\right\} R \sqrt{9 d \log \left(\frac{t}{\delta}\right)} s_i(t) .
\end{equation*}
Set $v_t = R \sqrt{9 d \log \left(\frac{t}{\delta}\right)}$. 
We have with probability at least $1 - t^{-2}$, 
\begin{equation*}
\begin{aligned}
    \left|b_i(t)^\top \tilde{\mu}(t)-b_i(t)^\top \hat{\mu}(t)\right| &\leq v_t\left\|b_i(t)\right\|_{B(t)^{-1}} \left\|\frac{1}{v_t} B(t)^{1/2}\left(\mut(t) - \muhat(t)\right)\right\|_{2} \\
    &= v_t s_i(t) \left\|\zeta\right\|_2 \leq v_t s_i(t) \sqrt{4 d \log t} 
\end{aligned}
\end{equation*}
where $\zeta$ denotes a standard normal random vector. 
Alternatively, we can bound $\left|\theta_i(t) - b_i(t)^\top \muhat(t)\right|$ for every $i$ by Gaussian concentration inequality and obtain 
\begin{equation*}
    \left|\theta_i(t)-b_i(t)^T \hat{\mu}(t)\right| \leq \sqrt{4 \log (N t)} s_i(t)
\end{equation*}
with probability at least $1 - 1/(N t^2)$. Taking a union bound, we obtain that $E^{\theta}(t)$ holds with probability at least $1-t^{-2}$. 

We call an arm $i$ is saturated at time $t$, if $\Delta_i(t) > g_t s_i(t)$, and unsaturated otherwise. 
Under both $E^{\theta}(t)$ and $E^{\mu}(t)$, by triangle inequality, we have
\begin{equation*}
    b_i(t)^\top \mu - g_t s_i(t) \leq \theta_i(t) \leq b_i(t)^\top \mu + g_t s_i(t).
\end{equation*}
Let $\bar{a}(t)$ denote the unsaturated arm with smallest $s_i(t)$, i.e.
$$
\bar{a}(t)=\arg \min _{i \notin C(t)} s_i(t).
$$
Combining with the fact that $\theta_{a(t)}(t) \geq \theta_i(t)$ for all $i$, we have 
$$
\begin{aligned}
\Delta_{a(t)}(t)= & \Delta_{\bar{a}(t)}(t)+\left(b_{\bar{a}(t)}(t)^\top \mu-b_{a(t)}(t)^\top \mu\right) \\
\leq & \Delta_{\bar{a}(t)}(t)+\left(\theta_{\bar{a}(t)}(t)-\theta_{a(t)}(t)\right) +g_t s_{t, \bar{a}(t)}+g_t s_{a(t)}(t) \\
\leq & \Delta_{\bar{a}(t)}(t)+g_t s_{t, \bar{a}(t)}+g_t s_{a(t)}(t) \\
\leq & g_t s_{t, \bar{a}}(t)+g_t s_{t, \bar{a}(t)}+g_t s_{a(t)}(t)
\end{aligned}
$$

Using anti-concentration of Gaussian random variable $\theta_{a^*(t)}$, one has 
$$
\Pr\left(\theta_{a^*(t)}(t)>b_{a^*(t)}(t)^\top \mu \mid \mathcal{F}_{t-1}\right) \geq \frac{1}{4e\sqrt{\pi}}. 
$$


Since $E^{\mu}(t)$ happens with high probability, we only need to focus on $\operatorname{regret}^{\prime}(t)=\operatorname{regret}(t) \cdot \mathbbm{1}\left(E^\mu(t)\right)$. 
Under $E^\mu(t)$, every arm is close to its expectation. 













\section{Action-centered contextual bandit}

\begin{table}[h]
    \centering
    \renewcommand{\arraystretch}{1.2} % Adjust row height
    \begin{tabular}{c p{12cm}}
        \hline
        \textbf{Notations} & \textbf{Description} \\ 
        \hline
        $\bar{s}_t \in \R^{d'}$ & Context vectors.\\
        $s_{t,a_t} \in \R^d$ & State vector, a function of $a_t$ and $\bar{s}_t$.\\
        $\calH_{t-1}$ & History of arms played, states and rewards, i.e., $$\mathcal{H}_{t-1}=\left\{a_\tau, \bar{s}_\tau, r_\tau\left(\bar{s}_\tau, a_\tau\right), i=1, \ldots, N, \tau=1, \ldots, t-1\right\}.$$\\
        $\pi(a,t)$ & Probability of playing arm $a$ at time $t$. \\
        $\hat{r}_t(\bar{a}_t)$ & $\left(I\left(a_t>0\right)-\pi_t\right) r_t\left(a_t\right)$. \\
        $\hat{b}$ & $\sum_{t=1}^T\left(I\left(a_t>0\right)-\pi_t\right) s_{t, \bar{a}_t} \hat{r}_t\left(a_t\right)$.\\
        $B$ & $I+\sum_{t=1}^T \pi_t\left(1-\pi_t\right) s_{t, \bar{a}_t} s_{t, \bar{a}_t}^\top$. \\
        $z_{t,a}$ & $\sqrt{s_{t, a}^\top B(t)^{-1} s_{t, a}}$.  \\
        $\ell(T)$ & $R \sqrt{d \log \left(T^3\right) \log (1 / \delta)}+1$. \\
        $v$ & $R \sqrt{\frac{24}{\epsilon} d \log (1 / \delta)}$.\\
        $g(T)$ & $\sqrt{4 d \log (T d)} v+\ell(T)$. \\
        $\theta', \tilde{\theta}$ & Random samples from $\mathcal{N}\left(\hat{\theta}, v^2 B^{-1}\right)$. \\
        $E^{\mu}(t)$ & The event that for all $\bar{a} \in [N]$, $\left|s_{t, \bar{a}}^\top \hat{\theta}_t-s_{t, \bar{a}}^\top \theta\right| \leq \ell(T) z_{t, \bar{a}}$. \\
        $E^{\theta}(t), E^{\theta}_0(t)$ & The event that $\left|s_{t, \bar{a}}^T \tilde{\theta}_t-s_{t, \bar{a}}^T \hat{\theta}_t\right| \leq \sqrt{4 d \log (T d)} v z_{t, \bar{a}}$, for all $\bar{a} \in [N]$\\
        \hline
    \end{tabular}
    \caption{Notation table}
    \label{tab:notation3}
\end{table}

Consider a contextual bandit with a baseline (zero) action and $N$ non-baseline arms. At each time $t$, a context vector $\bar{s}_t \in \R^{d'}$ is observed, an action $a_t \in \{0,1,\cdots, N\}$ is chosen, and a reward $r_t(a_t)$ is observed. The state vector $s_{t,a_t}$ is a function of $a_t$ and $\bar{s}_t$. Assume that the context vectors are chosen by an adversary on the basis of the history $\calH_{t-1}$ of arms $a_\tau$ played, states $\bar{s}_{\tau}$, and rewards $r_\tau(\bar{s}_{\tau}, a_\tau)$ received up to time $t-1$, i.e.
\begin{equation*}
    \calH_{t-1} = \{a_\tau, \bar{s}_\tau, r_\tau(\bar{s}_\tau, a_\tau), i\in[N], \tau \in [t-1]\}. 
\end{equation*}

The nice thing of being in a bandit setting is that we have access to the assignment probabilities and we can avoid making assumptions required in the causal inference setting. 
The conditional expectation of $\left(I\left(a_t>0\right)-\pi_t\right) r_t\left(a_t\right)$ is given by
\begin{equation*}
    \begin{aligned}
\mathbb{E}\left[\left(I\left(a_t>0\right)-\pi_t\right) r_t\left(a_t\right) \mid \mathcal{H}_{t-1}, \bar{a}_t, \bar{s}_t\right] & =\pi_t\left(1-\pi_t\right) r_t(\bar{a})-\left(1-\pi_t\right) \pi_t r_t(0) \\
& =\pi_t\left(1-\pi_t\right)\left(r_t\left(\bar{a}_t\right)-r_t(0)\right)
\end{aligned}
\end{equation*}
This observation leads to the optimization problem
\begin{equation*}
\begin{aligned}
\sum_{t=1}^T & \pi_t\left(1-\pi_t\right)\left(\hat{r}_t\left(\bar{a}_t\right) /\left(\pi_t\left(1-\pi_t\right)\right)-\theta^\top s_{t, \bar{a}_t}\right)^2+\|\theta\|_2^2 \\
& =\sum_{t=1}^T \left[\frac{\left(\hat{r}_t\left(\bar{a}_t\right)\right)^2}{\pi_t\left(1-\pi_t\right)}-2 \hat{r}_t\left(\bar{a}_t\right) \theta^\top s_{t, \bar{a}_t}+\pi_t\left(1-\pi_t\right)\left(\theta^\top s_{t, \bar{a}_t}\right)^2\right] + \|\theta\|_2^2 \\
& =c-2 \theta^\top \hat{b}+\theta^\top B \theta+\|\theta\|_2^2.
\end{aligned}
\end{equation*}
The selection probability $\pi_t$ is given by
\begin{equation*}
    \pi_t=\mathbb{P}\left(a_t>0\right)=\max \left(\pi_{\min }, \min \left(\pi_{\max }, \mathbb{P}\left(s_{t, \bar{a}}^\top \tilde{\theta}>0\right)\right)\right).
\end{equation*}
We analyze the regret 
\begin{equation*}
    \calR(T) = \sum_{t=1}^T \left(\pi_t^* s_{t, \bar{a}_t^*}^\top \theta-\pi_t s_{t, \bar{a}_t}^\top \theta\right) \leq \underbrace{\sum_{t=1}^T\left(\pi_t^*-\pi_t\right)\left(s_{t, \bar{a}_t}^\top \theta\right)}_I+\underbrace{\sum_{t=1}^T\left(s_{t, \bar{a}_t^*}^\top \theta-s_{t, \bar{a}_t}^\top \theta\right)}_{II}
\end{equation*}




\section{RoME}

There are some notational abusing issues. The expectation $\mathbb{E}[\cdot \mid s, \bar{a}]$ should be read as shorthand for:
$$
\mathbb{E}\left[\cdot \mid S_{i, t}=s, A_{i, t} \in\{0, \bar{a}\}, H_{i, t}\right]
$$

Compared to prior literature, RoME further consider subject-specific and time-specific random effects.
The reward is modeled as 
$$
R_{i, t}=g_t\left(S_{i, t}\right)+x\left(S_{i, t}, A_{i, t}\right)^{\top} \theta_{i, t} \cdot \mathbf{1}\left\{A_{i, t}>0\right\}+\epsilon_{i, t},
$$
where 
$$
\theta_{i, t}=\theta_{\text {shared }}+\theta_i^{\text {user }}+\theta_t^{\text {time }}.
$$
Due to the staged recruitment scheme assumption, there are $K$ users and $K$ time slots. 



\section{Non-stationary Bandits}

A stationary action model might make the historical data void or hard to justify any theoretical novelty. 
There are a few more flexible models. 










\section{Laplacian-Regularized Graph Bandits}

Consider a linear bandit with $m$ arms and $n$ users. Each arm is described by a feature vector $\vx$, each user is described by $\vtheta$. The payoff is given by
$$
y_t=\mathbf{x}_t^T \boldsymbol{\theta}_{i_t}+\eta_t.
$$
To estimate the paramter $\mTheta$ at time $t$, make use of 
$$
\hat{\boldsymbol{\Theta}}_t=\arg \min _{\boldsymbol{\Theta} \in \mathbb{R}^{n \times d}} \sum_{i=1}^n \sum_{\tau \in \mathcal{T}_{i, t}}\left(\mathbf{x}_\tau^T \boldsymbol{\theta}_i-y_\tau\right)^2+\alpha \operatorname{tr}\left(\boldsymbol{\Theta}^T \mathcal{L} \boldsymbol{\Theta}\right)
$$








\section{Semi-parametric inference based on
adaptively collected data}

$$
y=g\left(\left\langle x, \theta^*\right\rangle+h^*(z)\right)+\varepsilon,
$$
Here $g$ is a known link function. 
We observe a collection of $n$ samples of the form $\left(x_i, y_i, z_i\right)$ for $i=1, \ldots, n$.





\section{Contextual Linear Bandits under Noisy Features}

At each time $t$, the true feature $z_{a,t} \in \R^d$ for an arm $a$ in the set of arms is generated randomly encoding context, and the mean reward is $z_{a,t}^\top \theta^\star$ where $\theta^\star \in \R^d$ is a latent parameter. An agent can only observe a noisy feature vector $x_{a,t} \in \R^d$ rather than $z_{a,t}$, which is defined as $x_{a,t} = (z_{a,t} + \varepsilon_{a,t}) \circ m_{a,t}$. 



\section{Posterior Sampling}

Assume that $Y_{1:T} \sim p^\star$, where $p^\star$ is exchangeable. 
%This is a strong assumption, since even independent does not implies exchangeability. 
This assumption is basically i.i.d., except that the authors want to make use of a sequential model or something. 
Let $\pi$ denote an action selection algorithm. 
Let the per-user regret be 
$$
    \Delta\left(\pi ; p^*\right):=\mathbb{E}_{p^*, \pi}\left[\max _{a \in \mathcal{A}^{\text {new }}}\left\{\frac{1}{T} \sum_{t=1}^T R\left(Y_t^{(a)}\right)\right\}-\frac{1}{T} \sum_{t=1}^T R\left(Y_t^{\left(A_t\right)}\right)\right] .
$$
This definition relies on the exchangeability assumption. 
Moreover, this is a context-free setting. 
Without any additional context, exchangeability seems reasonable. 

They also have a contextual version. 
In this setting, the contexts are drawn i.i.d.~from an unknown distribution $P_X$. 
The pairs $(X_1, Y_1), \cdots, (X_T, Y_T)$ are assumed to be exchangeable. 




\section{Noisy Contextual Linear Bandit}

Here, we consider a more general setting. 
Suppose that $Y_1, Y_2, \cdots, Y_T$ follows certain distribution. 
%Let us consider the simplest setting first.
%\begin{assumption}
%    $(S_1, Y_1), \cdots, (S_T, Y_T)$ are exchangeable.  
%\end{assumption}
Intuitively, we would like to learn the distribution $Y_T \mid S_{1:T}$, so that we can sample from this distribution and use it as the current context. 

The first step is to investigate the log-likelihood of the data under the fitted distribution $p_{\beta}$ vesus the ground truth $p^\star$. 
Suppose that $Y_1, Y_2, \cdots, Y_T$ follows $p^\star$. 
Our goal is to bound
\begin{equation*}
    \E \sup_{g: \|g\|_{\infty} \leq 1} \left\{\E \left[g(Y_{1:T}) \mid S_{1:T}\right] - \E_{p_\beta} \left[ g(Y_{1:T}) \mid S_{1:T} \right] \right\} %\leq \frac{1}{\sqrt{2}} \left[L^\star(p_\beta) - L^\star(p^\star)\right]^{1/2}
\end{equation*}

\begin{proof}
    Fixing $S_{1:T} = s_{1:T}$ first. 
    \begin{equation*}
    \begin{aligned}
        &\sup_{g: \|g\|_{\infty} \leq 1} \left\{\E \left[g(Y_{1:T}) \mid S_{1:T} = s_{1:T}\right] - \E_{p_\beta} \left[ g(Y_{1:T}) \mid S_{1:T} = s_{1:T} \right] \right\} \\
        &~\stackrel{(i)}{\leq} d_{\mathrm{TV}}\left(\Pr_{p^\star}\left(Y_{1:T} \mid S_{1:T} = s_{1:T}\right), \Pr_{p_\beta}\left(Y_{1:T} \mid S_{1:T} = s_{1:T}\right)\right)\\
        &~\stackrel{(ii)}{\leq} \sqrt{\frac{1}{2} \KL\left(\Pr_{p^\star}\left(Y_{1:T} \mid S_{1:T} = s_{1:T}\right) \| \Pr_{p_\beta}\left(Y_{1:T} \mid S_{1:T} = s_{1:T}\right)\right)}
    \end{aligned}
    \end{equation*}
    where $(i)$ holds by the definition of total variation, $(ii)$ holds by Pinsker's inequality. 
    By Jensen's inequality, we have 
    \begin{equation*}
    \begin{aligned}
        &\E \sup_{g: \|g\|_{\infty} \leq 1} \left\{\E \left[g(Y_{1:T}) \mid S_{1:T}\right] - \E_{p_\beta} \left[ g(Y_{1:T}) \mid S_{1:T} \right] \right\} \\
        &~\leq \sqrt{\frac{1}{2} \E \KL\left(\Pr_{p^\star}\left(Y_{1:T} \mid S_{1:T} = s_{1:T}\right) \| \Pr_{p_\beta}\left(Y_{1:T} \mid S_{1:T} = s_{1:T}\right)\right)}\\
        &~= \sqrt{\frac{1}{2} \KL\left(\Pr_{p^\star}\left(Y_{1:T} \mid S_{1:T} \right) \| \Pr_{p_\beta}\left(Y_{1:T} \mid S_{1:T} \right)\right)}.
    \end{aligned}
    \end{equation*}
\end{proof}

%We have 
%\begin{equation*}
%\begin{aligned}
%    &\ell_{n}(p_\beta) - \ell_n(p^\star) \\
%    &~~= \E \left[- \sum_{t=1}^n \log p_{\beta} \left(Y_t \mid S_{1:t}\right)\right] - \E \left[-\sum_{t=1}^n \log p^\star \left(Y_t \mid S_{1:t}\right)\right] \\
%    &~~= \sum_{t=1}^n \E_{Y_t \mid S_{1:t} \sim p^\star} \left[-\log \left(\frac{p_{\beta}(Y_t \mid S_{1:t})}{p^\star(Y_t \mid S_{1:t})}\right)\right]\\
%    &~~= \sum_{t=1}^n \KL\left(\Pr_{p^\star}\left(Y_t \mid S_{1:t}\right) \| \Pr_{p_\beta}\left(Y_t \mid S_{1:t}\right)\right)
%\end{aligned}
%\end{equation*}
Next, we consider the reward given by
\begin{equation*}
    R(t, A_t) := \left\langle \thetastar, \phi(Y_t, A_t) \right\rangle + \eta_t,
\end{equation*}
where $Y_t$ is unobserved, $\phi$ is a known function. 
Define the conditional expectation
\begin{equation*}
    \psi^\star_a(S_{1:t}) := \E \left[\phi(Y_t, a) \mid S_{1:t}\right], \quad \psi^{(\beta)}_a(S_{1:t}) := \E_{p_\beta} \left[\phi(Y_t, a) \mid S_{1:t}\right].
\end{equation*}
Suppose that 
$$
\sup_{y} \max_{a \in \calA}\|\phi(y, a)\|_{\infty} \leq 1
$$
and 
$$
\sup_{y} \max_{a\in \calA} \|\phi(y, a)\|_2 \leq B
$$
By previous analysis, we have 
\begin{equation*}
    \left\|\psi^\star_a(s_{1:t}) - \psi^{(\beta)}_a(s_{1:t})\right\|_{\infty} \leq \sqrt{\frac{1}{2} \KL\left(\Pr_{p^\star}\left(Y_{1:t} \mid S_{1:t} = s_{1:t}\right) \| \Pr_{p_\beta}\left(Y_{1:t} \mid S_{1:t} = s_{1:t}\right)\right)} =: \sqrt{D_t}. 
\end{equation*}
Suppose that the oracle has access to $p^\star$ while we have access to an approximation $p_\beta$. 
First, we compare the reward obtained from running $\LinUCB$ under $p^\star$ and $p_\beta$. 
We aim to show that the regret can be well controlled.
Let $x_{t} := x_{t,A_t}$.
The Algorithm is given as follows:
\begin{algorithm}[H]
\begin{algorithmic}[1]
\REQUIRE $\lambda, \{\gamma_t\}_{t=1}^T$
\FOR{$t = 1$ to $T$}
\STATE $x_{t,a} \gets \psi_a^{(\beta)}(S_{1:t})$, $\Sigma_t \gets \lambda I + \sum_{\tau=1}^{t-1} x_\tau x_{\tau}$, $\thetahat_t \gets \Sigma_t^{-1} \sum_{\tau=1}^{t-1} r_{\tau} x_{\tau}$. 
\STATE Update
$$
\BALL_t \gets \left\{ \mu \mid \left(\thetahat_t - \mu\right)^\top \Sigma_t \left(\thetahat_t - \mu\right) \leq \gamma_t\right\}. 
$$
\STATE Choose action 
$$
A_t=\operatorname{argmax}_{a \in \calA} \max _{\mu \in \BALL_t} \mu^\top x_{t,a}
$$
with ties broken arbitrarily. 
\STATE Observe payoff $r_t \in [0, 1]$
\ENDFOR
\end{algorithmic}
\caption{LinUCB}
\label{alg:seq}
\end{algorithm}

Suppose that $\theta \in \BALL_t$.
Let $\tilde{\theta}_t \in \BALL_t$ denote the vector which maximizes the inner product $\mu^\top x_{t,\Astar_t}$, then
\begin{equation*}
    \thetahat_t^\top x_t = \max_{a \in \calA} \max_{\mu \in \BALL_t} \mu^\top x_{t,a} \geq \tilde{\theta}_t^\top x_{t, \Astar_t} \geq \theta^\top \xstar_t + \theta^\top (x_{t,\Astar_t} - \xstar_t). 
\end{equation*}
Define the regret at time $t$ to be 
\begin{equation*}
    \regret_t = r(t, \Astar_t) - r(t, A_t) = \theta^\top \xstar_t - \theta^\top x_t \leq (\thetahat_t - \theta)^\top x_t - \theta^\top \left(x_{t,\Astar_t} - \xstar_t\right). 
\end{equation*}
For any $\mu \in \BALL_t$, 
\begin{equation*}
    \left|(\mu - \thetahat_t)^\top x\right| \leq \sqrt{\gamma_t x^\top \Sigma_t^{-1} x}. 
\end{equation*}
It follows that 
\begin{equation*}
    \regret_t \leq \sqrt{\gamma_t x_t^\top \Sigma_t^{-1} x_t} - \theta^\top \left(x_{t,\Astar_t} - \xstar_t\right). 
\end{equation*}
Thus, 
\begin{equation*}
\begin{aligned}
     \sum_{t=1}^T \regret_t^2 &\leq 2\sum_{t=1}^T \gamma_t \min\left\{x_t^\top \Sigma_t^{-1} x_t, 1\right\} + 2\sum_{t=1}^T \left[\theta^\top \left(x_{t,\Astar_t} - \xstar_t\right)\right]^2\\
    &\leq 2\sum_{t=1}^T \gamma_t \min\left\{x_t^\top \Sigma_t^{-1} x_t, 1\right\} + 2 d \sum_{t=1}^T D_t. 
\end{aligned}
\end{equation*}
Following standard argument, we have 
\begin{equation*}
\begin{aligned}
    \sum_{t=1}^{T} \gamma_t \min\left\{x_t^\top \Sigma_t^{-1} x_t, 1\right\} &\leq 2\gamma_T \sum_{t=1}^T \log\left(1 + x_t^\top \Sigma_t^{-1} x_t\right) \\
    &\leq 4\gamma_{T} \log\left(\det \Sigma_{T} / \det \Sigma_1 \right) \\
    &= 4 \gamma_T d \log\left(1+\frac{T B^2}{d \lambda}\right).
\end{aligned}
\end{equation*}
Hence, 
\begin{equation*}
\begin{aligned}
     \sum_{t=1}^T \regret_t^2 
    &\leq 8 \gamma_T d \log\left(1+\frac{T B^2}{d \lambda}\right) + 2 d \sum_{t=1}^T D_t
\end{aligned}
\end{equation*}
and we have 
\begin{equation*}
    \sum_{t=1}^T \regret_t \leq \sqrt{T \sum_{t=1}^T \regret_t^2} \leq \sqrt{8T\gamma_T d \log \left(1 + \frac{T B^2}{d \lambda}\right) + 2dT \sum_{t=1}^T D_t}. 
\end{equation*}
It remains to choose a sequence of suitable $\{\gamma_t\}_{t=1}^T$ so that we have $\theta \in \BALL_t$ for all $t \in [T]$ with high probability.   

Let $\xstar_{t,a} := \psi^\star_a(S_{1:t})$.
At time $t$, we have  
$$
r_t = \theta^\top x_t + \theta^\top \left(\xstar_{t,A_t} - x_t\right) + \underbrace{\theta^\top \left(\xstar_{t,A_t} - \phi(Y_t, A_t)\right)}_{\varepsilon_t} + \eta_t.
$$
It follows that 
\begin{equation*}
\begin{aligned}
    \thetahat_t - \theta &= \Sigma_t^{-1} \sum_{\tau=1}^{t-1} r_{\tau} x_{\tau} - \theta \\
    &= \left[\Sigma_t^{-1} \left(\sum_{\tau=1}^{t-1} x_{\tau} x_{\tau}^\top\right) - 1\right] \theta + \Sigma_t^{-1} \sum_{\tau=1}^{t-1} x_{\tau} (\xstar_{\tau,A_\tau} - x_\tau)^\top \theta + \Sigma_t^{-1} \sum_{\tau=1}^{t-1} x_{\tau} (\varepsilon_\tau + \eta_\tau)\\
    &= -\lambda \Sigma_t^{-1} \theta + \Sigma_t^{-1} \sum_{\tau=1}^{t-1} x_{\tau} (\xstar_{\tau,A_\tau} - x_\tau)^\top \theta + \Sigma_t^{-1} \sum_{\tau=1}^{t-1} x_{\tau} (\varepsilon_\tau + \eta_\tau)\\
\end{aligned}
\end{equation*}
To ensure that $\varepsilon_t$ is a martingale difference sequence. 
Assume that the distribution of $Y_t$ is independent of $(A_1^{t-1}, r_{1}^{t-1})$ conditioned on $S_1^{t}$. 
%This assumption is reasonable, since if we follow the noisy feature setting, $Y_t$ would be independent of the previous observations. 
Under this assumption,
\begin{equation*}
    \E \left[\phi(Y_t, a) \mid A_{1:t-1}, r_{1:t-1}, S_{1:t}\right] = \xstar_{t,a}
\end{equation*}
Since $A_t$ is a function of $(A_{1:t-1}, r_{1:t-1}, S_{1:t})$, for any $a \in \calA$
\begin{equation*}
    \E \left[\phi(Y_t, a) \mid A_{1:t}, r_{1:t-1}, S_{1:t}\right] = \E \left[\phi(Y_t, a) \mid A_{1:t-1}, r_{1:t-1}, S_{1:t}\right] = \xstar_{t,a},
\end{equation*}
taking $a = A_t$ yields the desired result for $\varepsilon_t$.

The only term different from the vanilla $\LinUCB$ is that we have an extra term 
$$
\Sigma_t^{-1} \sum_{\tau=1}^{t-1} x_{\tau} (\xstar_{\tau,A_\tau} - x_\tau)^\top \theta.
$$
Using the crude bound
\begin{equation*}
    \left|(\xstar_{\tau,A_\tau} - x_\tau)^\top \theta\right| \leq \left\|\xstar_{\tau,A_\tau} - x_\tau\right\|_2 \leq \sqrt{d D_t} \leq \sqrt{d D_{\max}} ,
\end{equation*}
it follows that 
\begin{equation*}
\begin{aligned}
    &\left|\left(\sum_{\tau=1}^{t-1} x_{\tau}^\top (\xstar_{\tau,A_\tau} - x_\tau)^\top \theta\right) \Sigma_t^{-1} \left(\sum_{\tau=1}^{t-1} x_{\tau} (\xstar_{\tau,A_\tau} - x_\tau)^\top \theta\right)\right| \\ 
    &~\leq d D_{\max} \left|\left(\sum_{\tau=1}^{t-1} x_{\tau}\right)^\top \Sigma_t^{-1} \left(\sum_{\tau=1}^{t-1} x_{\tau}\right) \right|.
\end{aligned}
\end{equation*}
We note that 
\begin{equation*}
\begin{aligned}
    \left|\left(\sum_{\tau=1}^{t-1} x_{\tau}\right)^\top \Sigma_t^{-1} \left(\sum_{\tau=1}^{t-1} x_{\tau}\right) \right| &= \left\|\sum_{\tau=1}^{t-1} \Sigma_{t}^{-1/2} x_\tau\right\|_2^2\\
    &\leq (t-1) \sum_{\tau=1}^{t-1} x^\top_\tau \Sigma_{t}^{-1} x_\tau\\
    &= (t-1) \tr\left(\Sigma_t^{-1} \sum_{\tau=1}^{t-1} x_\tau x^\top_{\tau}\right)\\
    &= d(t-1) - \lambda (t-1) \tr\left(\Sigma^{-1}_t\right) \leq d(t-1). 
\end{aligned}
\end{equation*}

Therefore, with probability at least $1 - \delta_{t}$, 
\begin{equation*}
\begin{aligned}
    &\left\|\thetahat_t - \theta\right\|_{\Sigma_t^{-1}} \\
    &~\leq \sqrt{\lambda} + (\sigma_{\eta} + \sigma_{\varepsilon})\sqrt{2 \log (\det(\Sigma_t)\det(\Sigma_1)^{-1} / \delta_t)} + d\sqrt{D_{\max}(t-1)} \\
    &~\leq \sqrt{\lambda} + (\sigma_{\eta} + \sigma_{\varepsilon})\sqrt{2 \log \left[\left(1 + \frac{t B^2}{d \lambda}\right)^d/\delta_t\right]} + d\sqrt{D_{\max}(t-1)}. 
\end{aligned}
\end{equation*}
It suffices to set $\delta_t := \delta (3/\pi^2)/t^2$.
Hence, we can take 
\begin{equation*}
    \gamma_t := \underbrace{3\lambda_t + 6(\sigma_{\eta} + \sigma_{\varepsilon})^2 \log \left[4 t^2\left(1 + \frac{t B^2}{d \lambda}\right)^d/\delta\right]}_{\gamma^{(0)}_t} + 3d^2 D_{\max} t.
\end{equation*}
It follows that $\sum_{t=1}^T \regret_t$ is bounded by
\begin{equation*}
    \sqrt{2dT\sum_{t=1}^T D_t} + T D_{\max}^{1/2} \sqrt{24 d^3 \log \left(1 + \frac{T B^2}{d \lambda}\right)} + \sqrt{8 \gamma^{(0)}_T T d \log\left(1 + \frac{T B^2}{d \lambda}\right)} . 
\end{equation*}
%Consider the decomposition
%\begin{equation*}
%    R(t,A_t) = \left\langle \thetastar, \phi(Y_t, A_t) - \psi_{A_t}(S_1^{t}) \right\rangle  + \left\langle \thetastar, \psi_{A_t}(S_1^t) - \psihat_{A_t}(S_1^{t}) \right\rangle + \left\langle \thetastar, \psihat_{A_t}(S_1^{t}) \right\rangle + \eta_t.
%\end{equation*}
%Due to the inherent randomness, $\left\langle \thetastar, \phi(Y_t, A_t) - \psi_{A_t}(S_1^{t}) \right\rangle$ is a term cannot be controlled. 
%We will compare the regret with the oracle, which chooses action
%\begin{equation*}
%    \astar_t  = \arg \max_{a \in \calA} \left\langle \thetastar, \psi_a(S_1^t) \right\rangle
%\end{equation*}
%Define the regret at time $t$ to be
%\begin{equation*}
%    \regret_t := R(t, \astar_t) - R(t, a_t).
%\end{equation*}

%We use the historical data $\left(Y^{(0)}_1, \cdots, Y^{(0)}_{T_0}\right)$ to fit a working distribution $p_\beta(\cdot \mid S_{1}^t)$.
%Consider the difference
%\begin{equation*}
%    \psi_{A_t}(S_1^t) - \psihat_{A_t}(S_1^{t}) = \E \left[\phi(Y_t, a) \mid S_1^{t}\right] - \E_{p_\beta} \left[\phi(Y_t, a) \mid S_1^{t}\right]
%\end{equation*}













\section{LinUCB}

\begin{algorithm}[H]
\begin{algorithmic}[1]
\REQUIRE $\lambda, \beta_t$
\FOR{$t = 1$ to $T$}
\STATE Choose action $a_t=\operatorname{argmax}_{a \in \calA} \max _{\mu \in \operatorname{BALL}_t} \mu^\top a$ with ties broken arbitrarily. 
\STATE Observe payoff $r_t \in [0, 1]$
\STATE Update $\operatorname{BALL}_{t+1}$. 
\ENDFOR
\end{algorithmic}
\caption{LinUCB}
\label{alg:seq}
\end{algorithm}

The confidence ball is defined as 
\begin{equation*}
\operatorname{BALL}_{t} = \left\{ \mu \mid \left(\muhat_t - \mu\right)^\top \Sigma_t \left(\muhat_t - \mu\right) \leq \beta_t\right\},
\end{equation*}
where 
$$
\Sigma_t=\lambda I+\sum_{\tau=0}^{t-1} x_\tau x_\tau^{\top}, \text { with } \Sigma_0=\lambda I.
$$
We would want to understand the probability that $\mustar \in \operatorname{BALL}_{t}$
We have
\begin{equation*}
\begin{aligned}
    \muhat_{t} - \mustar &= \Sigma_t^{-1} \sum_{\tau=0}^{t-1} r_\tau x_\tau - \mustar\\
    &= \Sigma_t^{-1} \sum_{\tau=0}^{t-1} \left(\mustar \cdot x_{\tau} + \eta_{\tau}\right) x_\tau - \mustar\\
    &= - \lambda \Sigma_t^{-1} \mustar + \Sigma_t^{-1} \sum_{\tau=0}^{t-1} \eta_{\tau} x_{\tau}.
\end{aligned}
\end{equation*}
It follows that 
\begin{equation*}
\begin{aligned}
\sqrt{\left(\widehat{\mu}_t-\mu^{\star}\right)^{\top} \Sigma_t\left(\widehat{\mu}_t-\mu^{\star}\right)} & =\left\|\left(\Sigma_t\right)^{1 / 2}\left(\widehat{\mu}_t-\mu^{\star}\right)\right\| \\
& \leq\left\|\lambda \Sigma_t^{-1 / 2} \mu^{\star}\right\|+\left\|\Sigma_t^{-1 / 2} \sum_{\tau=0}^{t-1} \eta_\tau x_\tau\right\|
\end{aligned}
\end{equation*}


The regret at time $t$ is given by
$$
\operatorname{regret}_t=\mu^{\star} \cdot x^*-\mu^{\star} \cdot x_t.
$$
To bound the cumulative regret, it suffices to bound the sum of squares regret. 




\section{Explore First, Exploit Next}

Consider finitely many arms $a \in [K]$. Each is associated with an unknown probability distribution $v_a$ over $\R$. At each round $t \geq 1$, the player pulls the arm $A_t$ and get a real-valued reward $Y_t$ drawn independently at random according to the distribution $v_{A_t}$. A strategy $\psi$ associates an arm with the information gained in the past, possibly based on some auxiliary randomization; without loss of generality, this randomization is provided by $U_0, U_1, U_2, \cdots$. A strategy is a sequence $\psi = (\psi_{t})_{t \geq 0}$ of measurable functions, each of  which associates with the said past information, namely,
\begin{equation*}
    I_t = \left(U_0, Y_1, U_1, \cdots, Y_t, U_t\right),
\end{equation*}
an arm $\psi_t(I_t) = A_{t+1} \in [K]$. 

\appendix

\bibliographystyle{apalike}
\bibliography{rl}

\section{Martingale}

We will revisit some classical definitions and results here. 
The conditional expectation of $X$ given $\calF$ is any random variable $Y$ that has 
\begin{itemize}
    \item[(i)] $Y\in \calF$, i.e., is $\calF$ measurable
    \item[(ii)] for all $A \in \calF$, $\int_{A} X dP = \int_{A} Y dP$.  
\end{itemize}













\section{Martingale concentration inequalities}

\begin{lemma} \label{lem:self-norm}
    Let $(\calF_t)_{t\geq 0}$ be a filtration, $(m_t)_{t\geq 1}$ be an $\R^d$-valued stochastic process such that $m_t$ is $(\calF_{t-1})$-measurable, $(\eta_t)_{t\geq 1}$ be a real-valued martingale difference process such that $\eta_t$ is $(\calF_t)$-measurable. 
For $t \geq 0$, define $\xi_t = \sum_{\tau=1}^t m_\tau \eta_{\tau}$ and $M_t = I_d + \sum_{\tau=1}^t m_\tau m_\tau^\top$, where $I_d$ is the $d$-dimensional identity matrix. Assume $\eta_t$ is conditionally $R$-sub-Gaussian. Then for any $\delta > 0$, $t \geq 0$, with probability at least $1 - \delta$, 
\begin{equation*}
    \left\|\xi_t\right\|_{M_t^{-1}} \leq R \sqrt{d \log \left(\frac{t+1}{\delta}\right)},
\end{equation*}
where $\|\xi_t\|_{M_t^{-1}} = \sqrt{\xi_t^\top M_t^{-1} \xi_t}$. 
\end{lemma}
























\section{Information theory}























\section{Semiparametric Statistics}

The log-likelihood is denoted by 
\begin{equation*}
    \ell_{\theta}(x) = \log p_{\theta}(x), \quad \text{for } x \in \calX.
\end{equation*}
Suppose that $\theta \mapsto p_{\theta}(x)$ is differentiable at $\theta$ for all $x \in \calX$, we call $\dot{\ell}_\theta(x)$ the score function. 
The Fisher information at $\theta$ is defined as 
$$
I(\theta) \equiv I_\theta=\mathbb{E}\left[\dot{\ell}_\theta(X) \dot{\ell}_\theta(X)^{\top}\right], \quad \text { where } X \sim P_\theta.
$$

An important result on the geometry of influence functions is given as below
\begin{theorem}
    Under some regular conditions, let the parameter of interest be $\psi(\theta)$, a $q$-dimensional function of the $k$-dimensional parameter $\theta (q < k)$ such that
    \begin{equation*}
        \frac{\partial}{\partial \theta} \psi(\theta) \equiv \dot{\psi}_\theta,
    \end{equation*}
    exists at the ground truth $\theta_0$. 
    Also let $T_n$ be an asymptotically linear estimator with influence function $\varphi(X)$ such that $\mathbb{E}_\theta\left[\varphi(X)^{\top} \varphi(X)\right]$ exists at $\theta_0$. Then, if $T_n$ is regular, this will imply that
$$
\mathbb{E}\left[\varphi(X) \dot{\ell}_{\theta_0}(X)^{\top}\right]=\dot{\psi}_{\theta_0}
$$
\end{theorem}
Suppose that $\theta \equiv (\psi, \eta)$. We can think of this as $\psi(\theta) \equiv \psi$ so that $\Gamma(\theta) \equiv \dot{\psi}_\theta=\left(I_q, 0_{q \times(k-q)}\right)$ is a $q \times k$ matrix.

Let $X \sim P$ be a random variable. Let $\calH$ be the Hilbert space of mean-zero $q$-dimensional measurable functions of $X$ with finite second moments and the inner product
\begin{equation*}
    \left\langle h_1, h_2 \right\rangle := \E \left[h_1^\top h_2\right]
\end{equation*}
Let $v$ be an $r$-dimensional random function with mean zero and $\|v\| \leq \infty$. The linear subspace $\calU$ spanned by $v$ is given by
\begin{equation*}
    \mathcal{U}=\left\{B_{q \times r} v(X) \text { : where } B \in \mathbb{R}^{q \times r} \text { is any arbitrary matrix }\right\} .
\end{equation*}
The nuisance tangent space is the linear subspace spanned by the nuisance score vector $\dot{\ell}_{\theta_0}^{(2)}(X)$, i.e.,
$$
\Lambda:=\left\{B \dot{\ell}_{\theta_0}^{(2)}(X): \text { where } B \in \mathbb{R}^{q \times(k-q)} \text { is any arbitrary matrix }\right\} .
$$

To project an arbitrary element $h(X) \in \calH$ onto $\calU$, the projection must satisfy
$$
\left\langle h-B_0 v, B v\right\rangle=\mathbb{E}\left[\left\{h(X)-B_0 v(X)\right\}^{\top} B v(X)\right]=0 \quad \text { for all } B \in \mathbb{R}^{q \times r},
$$
implying
$$
B_0=\mathbb{E}\left[h(X) v(X)^{\top}\right]\left\{\mathbb{E}\left[v(X) v(X)^{\top}\right]\right\}^{-1}.
$$
Hence, the unique projection of $h(X) \in \mathcal{H}$ onto $\mathcal{U}$ is
$$
\Pi(h \mid \mathcal{U})=\mathbb{E}\left[h(X) v(X)^{\top}\right]\left\{\mathbb{E}\left[v(X) v(X)^{\top}\right]\right\}^{-1} v
$$

The efficient score is the residual of the score vector with respect to the parameter of interest after projecting it onto the nuisance tangent space, i.e.,
$$
\dot{\ell}_{\theta_0}^{\text {eff }}:=\dot{\ell}_{\theta_0}^{(1)}-\Pi\left(\dot{\ell}_{\theta_0}^{(1)} \mid \Lambda\right) .
$$
By the above projection formula, we have
$$
\Pi\left(\dot{\ell}_{\theta_0}^{(1)} \mid \Lambda\right)=\mathbb{E}\left[\dot{\ell}_{\theta_0}^{(1)}(X) \dot{\ell}_{\theta_0}^{(2)}(X)^{\top}\right]\left\{\mathbb{E}\left[\dot{\ell}_{\theta_0}^{(2)}(X) \dot{\ell}_{\theta_0}^{(2)}(X)^{\top}\right]\right\}^{-1} \dot{\ell}_{\theta_0}^{(2)} .
$$
When the parameter $\theta$ can be partitioned as $(\psi, \eta)$, where $\psi$ is the parameter of interest and $\eta$ is the nuisance parameter, then the efficient influence function can be written as 
$$
\varphi^{\mathrm{eff}}=\left\{\mathbb{E}\left[\dot{\ell}_{\theta_0}^{\mathrm{eff}}(X) \dot{\ell}_{\theta_0}^{\mathrm{eff}}(X)^{\top}\right]\right\}^{-1} \dot{\ell}_{\theta_0}^{\mathrm{eff}} .
$$

\end{document}

